\documentclass[11pt]{article}
\usepackage[sc]{mathpazo} %Like Palatino with extensive math support
\usepackage{fullpage}
\usepackage[authoryear,sectionbib,sort]{natbib}
\linespread{1.7}
\usepackage[utf8]{inputenc}
\usepackage{lineno}

%%%%%%%%%%%%%%%%%%%%%
% LaTeX packages
%%%%%%%%%%%%%%%%%%%%%
% Please be sparing in your use of additional LaTeX packages, and
% upload any required style files to Editorial Manager with the file
% type "LaTeX ancillary files (.sty, .bst)."

%%%%%%%%%%%%%%%%%%%%%
% Line numbering
%%%%%%%%%%%%%%%%%%%%%
\usepackage{lineno}
% Please use line numbering with your initial submission and
% subsequent revisions. After acceptance, please comment out 
% the commands \usepackage{lineno}, \linenumbers{} 
% and \modulolinenumbers[3] below.

\title{Predator phylogenetic diversity decreases predation rate via antagonistic interactions}

%%%%%%%%%%%%%%%%%%%%%
% Authorship
%%%%%%%%%%%%%%%%%%%%%
% Please remove authorship information while your paper is under review,
% unless you wish to waive your anonymity under double-blind review. 
% Remember to uncomment the information after acceptance.

\author{A. Andrew M. MacDonald$^{1,\ast}$ \\ 
Gustavo Q. Romero$^{1,\dag}$ \\ 
Diane S. Srivastava$^{2,\ddag}$}

\date{}

\begin{document}

\maketitle

\noindent{}1. Biodiversity Research Centre & Department of Zoology, University of British Columbia, Vancouver, British Columbia, V6T1Z4, Canada;

\noindent{}2. Departamento de Biologia Animal, Instituto de Biologia, Universidade Estadual de Campinas (UNICAMP), CP 6109, CEP 13083-970 Campinas, S\~{a}o Paulo, Brazil.

\noindent{}$\ast$ Corresponding author; e-mail: macdonal@zoology.ubc.ca

\noindent{}$\ddag$ ORCIDs: MacDonald, orcid.org/0000-0003-1162-169X; .

\bigskip

\textit{Manuscript elements}: Figure~1, figure~2, table~1, online
appendices~A and B (including figure~A1 and figure~A2). %Figure~2 is to
% print in color.

\bigskip

\textit{Keywords}: Predators, food webs, phylogenies.

\bigskip

\textit{Manuscript type}: Article. 
% Or e-article, note, e-note, natural history miscellany,
% e-natural history miscellany, comment, reply, symposium, or
% countdown to 150.

\bigskip

\noindent{\footnotesize Prepared using the suggested \LaTeX{} 
template for \textit{Am.\ Nat.}}

\linenumbers{}
\modulolinenumbers[3]

\newpage{}

\section*{Abstract}

Predator assemblages can differ substantially in their top-down effects on
community composition and ecosystem function, but few studies have sought to
explain this variation in terms of the phylogenetic diversity (PD) of
predators. When predators show broad overlap in their fundamental niches, a
range of PD may be represented in local predator assemblages. In this case, if
distantly-related predators overlap less in their diet, then predator
assemblages with high PD should consume more of the prey community.
Alternatively, if distantly-related predators show more antagonistic
interactions, predator assemblages with high PD should consume less of the
prey community. Either effect of predator PD on prey mortality could have
cascading effects on the ecosystem functions mediated by prey. We examined
predator PD in macroinvertebrate food webs found in bromeliads, a natural
aquatic mesocosm. We use measures of predator PD to combine three datasets:
observations of predator distribution among bromeliads, experimental feeding
trials, and a manipulation of predator PD. We found that phylogenetic distance
does not predict differences in predator distribution, indicating that a range
of predator PD is found in nature. We did, however, find a tendency for
distantly-related predators to eat different prey, a prerequisite for
synergistic effects of predators on prey mortality. However, our manipulative
experiment showed that increasing predator PD reduced prey mortality,
reflecting antagonistic interactions among more distant predators. These
effects of PD on prey mortality did not translate into effects on ecosystem
function, as measured by rates of decomposition and nitrogen cycling. In
conclusion, the effects of predator PD on the bromeliad food web are primarily
determined by antagonistic predator-predator interactions, rather than habitat
distribution or diet overlap. This study illustrates the potential of a
phylogenetic community approach to understanding food webs.


\newpage{}

\section*{Introduction}

% The journal does not have numbered sections in the main portion of
% articles. Please refrain from using section references such as
% section~\ref{section:CountingOwlEggs}, and refer to sections by name
% (e.g. section ``Counting Owl Eggs'').


Predators can have strong top-down effects, both on community structure
and ecosystem processes \citealt{Estes2011}. The combined effect of
predator species on communities is often stronger or weaker than that
predicted from a study of those same species in isolation
\citealt{Sih1998a, Ives2005}. These non-additive effects occur when
predators interact with each other directly, or via their shared prey
species. For example, predators feed directly on each other (intra-guild
predation), consume the same prey (resource competition) or modify the
behaviour of prey or the other predator species
\citealt{Sih1998a, Griswold2006, Nystrom2001}. These non-additive effects
can be positive or negative. For example, prey may have an induced
defense against one predator which increases (negative non-additive
effect) or decreases (positive non-additive effect) the likelihood of
consumption by a second predator. While there are many possible
mechanisms underlying the effect of predator composition, we lack a
means of predicting \emph{a priori} the strength and direction of this
effect on community structure and ecosystem function.

The phylogenetic relationships among predators could provide a framework for combining different approaches to studyin predator-predator interactions, thus helping us make
predictions about combined effects of predators. A phylogenetic approach to species interactions
extends the measurement of species diversity to include the evolutionary
relationships between species. Relatedness may be a proxy for ecological similarity; very
similar species may compete strongly, and/or may interfere with each
other while very different species may not be able to occur in the same
patch. This approach was first used to interpret observations of
community structure, as ecologists interpreted nonrandom phylogenetic
structure (i.e.~under- or over- dispersion) as evidence for the
processes, such as habitat filtering or competition, which structure
communities \citealt{Webb2002, Cavender-Bares2009}. Recently, this
approach has been applied to manipulative experiments. For example, the
phylogenetic diversity of plant communities is a better predictor of
productivity than either species richness or diversity
\citealt[e.g.][]{Cadotte2009, Cadotte2008, Godoy2014}. In all cases, an
implicit assumption is that increased phylogenetic distance is
associated with increased ecological dissimilarity -- either in the form
of differences in species niches, interactions, or functional traits.
When this is true, high phylogenetic diversity should lead to
complementarity in resource use between species, resulting in increased
ecosystem functioning \citealt{Srivastava2012c}. 

Phylogenetic diversity may be a better predictor of species effects on ecosystem funcitioning than species identity along. For example, studies of
plants have shown that in both experimental \citealt{Cadotte2008} and
natural communities, ecosystem function is
positively related to the phylogenetic diversity of plants. Although
there have been many studies taking a phylogenetic approach to community
ecology and although predators have large effects on many communities,
the phylogenetic diversity of local predator assemblages has rarely been
measured \citealt{Bersier2008, Naisbit2011}. Many studies of phylogeny and
predator traits focus on whole clades, rather than local assemblages
(e.g. \emph{Anolis} lizards \citealt{Knouft2006}, warblers
\citealt{Bohning-Gaese2003}, tree boas \citealt{Henderson2013} and wasps
\citealt{Udriene2005}), making it difficult to connect these results to
predator effects at the scale of a local community. These clade specific
studies often find weak evidence for phylogenetic signal in ecologically
relevant traits. In contrast, studies at the level of the whole
biosphere \citealt{Gomez2010, Bersier2008} demonstrate that related
organisms often have similar interspecific interactions, i.e.~related
predators often consume similar prey. At the local scale, only a few
studies have examined how phylogeny may shape food webs
\citealt{Rezende2009, Cagnolo2011}; these observational studies found that
models containing both relatedness (either from taxonomic rank or
phylogenetic trees) and body size were better at predicting which
predator-prey interactions occurred than models with body size alone. As
observational studies, however, they cannot isolate if it is differences
in predator distribution or diet that leads to a phylogenetic signal in
predator-prey interactions, nor how these interactions affect the whole
community.

Can
phylogeny help us predict how predators will impact community composition and ecosystem functioning? Within a local community, the effect of predator species diversity will
depend on three factors: how predators are distributed among habitats,
how they interact with their prey, and how they interact with each other. To the extent that phylogenetic relationships are
correlated with these three factors they enable us to predict the impact
of predator diversity on communities. For instance, phylogeny could
constrain predator species co-occurrence if more distant phylogenetic
relatives have more distinct fundamental niches, whereas close relatives
are too similar to co-exist \citealt{Webb2002, Emerson2008}. When
predators do co-occur, phylogeny may correlate with their feeding
behavior, such that closely related predators consume similar prey. For
example, diet overlap (shared prey species between predators) will
depend on the feeding traits and nutritional requirements of predators
-- both of which may be phylogenetically conserved. If this is the case,
then predator assemblages with higher phylogenetic diversity will show a
greater range of prey consumed and therefore stronger top- down effects
\citealt{Finke2008a}. In some cases, predator diets may extend to include
other predators, leading to direct negative interactions such as
intraguild predation, which may also have a phylogenetic signal
\citealt{Pfennig2000}. To our knowledge, the relationship of phylogeny to
predator distribution, diet, and intraguild interactions has never been
investigated in a single study.

We tested for the effects of phylogenetic distance on distribution, diet
and interactions of predators living in a natural mesocosm: water
reservoirs found inside bromeliad leaves. Bromeliads (Bromeliaceae) are
flowering plants abundant in the Neotropics. Within this aquatic food
web, damselfly larvae (e.g. \emph{Leptagrion} spp.,
Odonata:Coenagrionidae) are important predators that dramatically reduce
insect colonization \citealt{Hammill2015} and emergence
\citealt{Starzomski2010}, and increase nutrient cycling \citealt{Ngai2006}.
In addition to damselfly larvae, other predators are also found in
bromeliads, including large predaceous fly larvae (Diptera: Tabanidae)
and predatory leeches (Hirudinae:Arhynchobdellida) (see Frank et al.
\citeyearpar{Frank2009}). Many bromeliads contain water and trapped,
terrestrial detritus which supplies nutrients for the bromeliad
\citealt{Reich2003a}. The small size of these habitats permits direct
manipulations of entire food webs, manipulations which would be
difficult in most natural systems. Predators have been shown to have
large top-down effects on ecosystem functions in bromelaids, including
nitrogen uptake by the plant \citealt{Ngai2006}, detrital decomposition
and $CO\textsubscript{2}$ flux \citealt{Atwood2014, Atwood2013}.

We tested for a relationship between the distribution, diet and
ecosystem effect of predators and their phylogenetic distance using
observations, lab feeding trials, and manipulative field experiments,
respectively. We observed predator distribution by dissecting a sample
of natural bromeliads. We quantified diet preferences in a series of
no-choice feeding trials. Ecosystem-level effects were measured with a
manipulative experiment, where predators were placed alone or in
combination within standardized communities. In each approach, we test
the hypothesis that greater phylogenetic distance correlates with
greater difference in predator impacts on the bromeliad community:

\begin{enumerate}
\def\labelenumi{\arabic{enumi}.}
\item
  \emph{Distributional similarity}: Closely related predators occur in
  the same habitat patch more frequently than less related predators.
  Alternatively, closely related species may never co-occur.
\item
  \emph{Diet similarity}: Closely related predators will eat similar
  prey at similar rates (in other words, there should be a correlation between the relatedness of two predators and their similarity of their diets). Alternatively, closely related species may have
  evolved different diets to facilitate coexistence.
\item
  \emph{Ecosystem-level effects}: We tested two sets of hypotheses about
  direct and indirect effects of predator combinations on ecosystems,
  predicting:

  \begin{enumerate}
  \def\labelenumii{(\alph{enumii})}
  % \tightlist
  \item
    Closely related predators will have similar effects on the
    community and ecosystem. This will occur if related predators have similar trophic
    interactions (e.g.~predation rate, diet similarity). Our
    single-species treatments allow us to assess the effect of each
    predator both on prey survival and on ecosystem functions.
  \item
    Predator assemblages with higher phylogenetic diversity will have
    synergistic (greater than additive) effects on prey consumption and
    associated ecosystem functions. This will occur if phylogenetic
    distance correlates with increasing trait difference, and if this
    trait difference in turn results in niche complementarity. However,
    at the extreme, different predators may consume each other, thus
    creating antagonistic (less than additive) effects on prey
    consumption. By comparing treatments with pairs of predators to
    treatments that received each predator alone, we are able to
    estimate additive and non-additive effects.
  \end{enumerate}
\end{enumerate}

% Please note that we prefer (\citealt{Xiao2015}) to \citep{Xiao2015},
% since \citep{} inserts a comma after "et al."

\section*{Methods}

The quick red fox jumps over the lazy brown dog. Furthermore, the quick 
brown fox jumps over the lazy red dog. As a result, the quick 
R\"{u}ppell's fox (\textit{Vulpes rueppellii}) jumps over the lazy 
golden retriever.

\subsection*{The quickness of the fox}

Nulla facilisi, despite the findings of \citet{LemKapEx07}. Etiam 
semper, orci sit amet facilisis interdum, tellus nunc consequat erat, 
quis viverra nisi diam ut metus. Pellentesque cursus, sapien malesuada 
euismod iaculis, mauris purus interdum diam, vel vestibulum justo enim 
vitae tellus. Nunc interdum lorem sit amet diam volutpat tristique. 
Quisque pulvinar ac metus commodo lacinia (\citealt{Ing11,Xiao2015}).  

\subsection*{The redness of the fox}

As \citet{Xiao2015} argued, phasellus porttitor eros et ante 
condimentum, eget facilisis orci condimentum. Nulla facilisi. Proin 
placerat elit blandit, euismod dolor nec, dapibus diam. Mauris posuere 
malesuada lacus, at elementum lacus auctor eu (fig~\ref{Fig:Jumps}A). 

\section*{Results}

Aenean pulvinar malesuada commodo (see \citealt{DavisEtAl2011}; 
table~\ref{Table:Okapi}). Sed aliquet mauris odio, in tristique dui 
egestas a. Etiam eu malesuada quam. Suspendisse tincidunt eu erat sit 
amet vulputate. Duis at arcu et nisl dictum mattis. Maecenas vel cursus 
ante. Cras eleifend elit nec velit sollicitudin fermentum in ac mauris. 
Pellentesque rutrum magna vel elit maximus hendrerit. All data are 
available in the Dryad Digital Repository (\citealt{CookEtAl2015}).

\subsection*{The height of the jump}

Aenean eu pellentesque quam (fig.~\ref{Fig:OkapiHorn}). Nam pellentesque 
augue eu finibus lacinia. Nullam nec justo vitae odio imperdiet rhoncus 
vitae vitae quam. Pellentesque porttitor metus et lectus ornare, ac 
cursus urna efficitur (fig~\ref{Fig:Jumps}B). 

\subsection*{The laziness of the dog}

Sed sit amet pharetra nisi (fig.~\ref{Fig:AnotherFigure}). Praesent 
quis dolor in dolor molestie cursus et ac nisi. Vestibulum ante purus, 
semper eget est vitae, vehicula ornare nisl. Morbi efficitur euismod 
enim, nec feugiat tellus cursus eget. Donec mauris nibh, volutpat 
vehicula viverra at, iaculis congue sem. Praesent eget erat rhoncus erat 
sollicitudin volutpat. 

\section*{Discussion}

Nam pulvinar lorem at lorem ultrices, vel accumsan massa feugiat 
(\citealt{Ing11}). Proin tristique velit eget lacus iaculis, in 
pellentesque nulla varius. Phasellus sodales est odio, eu pulvinar 
magna pellentesque eu. Sed ut lobortis eros. Aliquam eget metus turpis. 
Sed et convallis lectus, id tincidunt enim. In porta nibh ut lacus 
feugiat, non consequat orci rhoncus. Morbi blandit at augue nec tempor. 
Sed fringilla ipsum ut justo viverra, ut euismod nisi gravida.

Curabitur non posuere augue, id suscipit orci. Nunc luctus accumsan 
aliquam. Cras egestas turpis vitae nisl vulputate interdum. Donec 
pellentesque libero egestas tortor pharetra laoreet. Phasellus facilisis 
auctor ligula, eu sollicitudin mi sagittis non.

\section*{Conclusion}

Duis pharetra enim at libero cursus, eu commodo mi vestibulum. Nullam 
eget velit nec lectus viverra sodales. Suspendisse egestas, eros at 
dictum tincidunt, mi orci laoreet libero, eget rutrum sapien arcu 
blandit odio.

%%%%%%%%%%%%%%%%%%%%%
% Acknowledgments
%%%%%%%%%%%%%%%%%%%%%
% You are encouraged to remove the Acknowledgments section while
% your paper is under review (unless you wish to waive your anonymity
% under double-blind review) if the Acknowledgments reveal your
% identity. If you remove this section, you will need to add it back
% in to your final files after acceptance.

\section*{Acknowledgments}

OEC would like to thank the world. GHC is much indebted to 
the solar system. AQE was supported by a generous grant from 
the Milky Way (MW/01010/987654).

\newpage{}

\renewcommand{\thesection}{\Alph{section}}

\section*{Online Appendix A: Supplementary Figures}

% Subsection numbering is permitted (but by no means necessary) in 
% online appendices. Please note that if you have sections (thus Online
% Appendix A, B, and C), these will become three separate online PDFs.
% You may wish to consolidate these into one PDF (hence one section,
% divided into subsections as necessary). Please reset counters for
% each such section. 

\renewcommand{\theequation}{A\arabic{equation}}
% redefine the command that creates the equation number.
\renewcommand{\thetable}{A\arabic{table}}
\setcounter{equation}{0}  % reset counter 
\setcounter{figure}{0}
\setcounter{table}{0}

\subsection*{Fox--dog encounters through the ages}

The quick red fox jumps over the lazy brown dog. The quick red fox has 
always jumped over the lazy brown dog. The quick red fox began jumping 
over the lazy brown dog in the 19th century and has never ceased from so 
jumping, as we shall see in figure~\ref{Fig:Jumps}.

[Figure A1 goes here.]

[Figure A2 goes here.]

\newpage{}

\section*{Online Appendix B: Additional Methods}

\renewcommand{\theequation}{B\arabic{equation}}
% redefine the command that creates the equation number.
\setcounter{equation}{0}  % reset counter 
\renewcommand{\thetable}{B\arabic{table}}
\setcounter{figure}{0}
\setcounter{table}{0}

\subsection*{Measuring the height of fox jumps without a meterstick}

Pellentesque ac nibh placerat, luctus lectus non, elementum mauris. 
Morbi odio velit, eleifend ut hendrerit vitae, consequat sit amet 
nulla. Pellentesque porttitor vitae nisl quis tempus. Pellentesque 
habitant morbi tristique senectus et netus et malesuada fames ac 
turpis egestas. Praesent ut nisi odio. Vivamus vel lorem gravida 
odio molestie volutpat condimentum et arcu. 

\begin{equation}
{ \frac{1}{N_k-1} \sum \limits_{t=1}^{N_k} (M_{tjk} - \bar{M}_{jk})^2}
\end{equation}

\subsection*{Quantifying the brownness of the dog}

Pellentesque eu nulla odio (\citealt{Xiao2015,CookEtAl2015}). Nulla 
aliquam porta metus, quis malesuada orci faucibus quis. Suspendisse nunc 
magna, tristique sit amet sollicitudin nec, elementum et lacus. Sed 
vitae elementum mi. In hac habitasse platea dictumst. Etiam eu tortor 
elit. Sed ac tortor purus. Aliquam volutpat, odio sit amet posuere 
pretium, dolor ex interdum ante, sed luctus quam eros ac nulla. 

\begin{equation}
{ (\sum \limits_{p=1}^P {n_{sp}})^{-1}\sum \limits_{p=1}^P {n_{sp}Q_{p}}}
\end{equation}

\newpage{}

%%%%%%%%%%%%%%%%%%%%%
% Bibliography
%%%%%%%%%%%%%%%%%%%%%
% You can either type your references following the examples below, or
% compile your BiBTeX database and paste the contents of your .bbl file
% here. The amnatnat.bst style file should work for this---but please
% email the journal office at amnat at uchicago dot edu if you run into
% any hitches with it!
% The list below includes sample journal articles, book chapters, and
% Dryad references.

%\bibliographystyle{amnatnat}
\begin{thebibliography}{}

\bibitem[{Cook et~al.(2015)Cook, Collaborator, and Expert}]{CookEtAl2015}
Cook, O.~E., G.~H. Collaborator, and A.~Q. Expert. 2015.
\newblock Data from: Template and guidelines for using \LaTeX{} 
in \textit{The American Naturalist}.
\newblock American Naturalist, Dryad Digital Repository, 
http://dx.doi.org/10.5061/dryad.XYZAB.

\bibitem[{Davis et~al.(2011)Davis, Brakora, and Lee}]{DavisEtAl2011}
Davis, E.~B., K.~A. Brakora, and A.~H. Lee. 2011.
\newblock Evolution of ruminant headgear: a review.
\newblock Proceedings of the Royal Society B 278:2857--2865.

\bibitem[{Inglis et~al.(2011)Inglis, Roberts, Gardner, and Buckling}]{Ing11}
Inglis, R.~F., P.~G. Roberts, A.~Gardner, and A.~Buckling. 2011.
\newblock Spite and the scale of competition in \textit{Pseudomonas
  aeruginosa}.
\newblock American Naturalist 178:276--285.

\bibitem[{Lemod\`{e}le et~al.(2007)Lemodele, Kapitelschreiber, 
and Exemplar}]{LemKapEx07}
Lemod\`{e}le, P.-Q., A.~B. Kapitelschreiber, and C.~D.~E. Exemplar. 2007.
\newblock An exemplary instance of chapters in books.
\newblock Pages 231--245 \emph{in} J.-P. \'{E}crivain and M.~A. 
Term\'{e}szettud\'{o}s, eds. Inspiring Instances of Brilliant Writing. 
Truth Pudding Press, Fond du Lac, WI.

\bibitem[{Xiao et~al.(2015)Xiao, McGlinn, and White}]{Xiao2015}
Xiao, X., D.~J. McGlinn, and E.~P. White. 2015.
\newblock A strong test of the maximum entropy theory of ecology.
\newblock American Naturalist 185:E705--E80.

\end{thebibliography}

\newpage{}

\section*{Tables}
\renewcommand{\thetable}{\arabic{table}}
\setcounter{table}{0}

\begin{table}[h]
\caption{Animals in various cities with equations}
\label{Table:Okapi}
\centering
\begin{tabular}{llc}\hline
Animal    & City         & Equation \\ \hline
Dog       & Springfield  & $x+y=z$ \\
Fox       & Indianapolis & $2x+2y=2z$ \\
Okapi$^a$ & Chicago      & $x-y<z$ \\
Badger    & Madison      & $x+2y>z$ \\ \hline
\end{tabular}
\bigskip{}
\\
{\footnotesize Note: Table titles should be short. Further details 
should go in a `notes' area after the tabular environment, as shown 
here. $^a$ Okapis are not native to Chicago, but they are to be met with 
in both of the major Chicagoland zoos.}
\end{table}

\newpage{}

%%%%%%%%%%%%%%%%%%%%%
% Figure legends
%%%%%%%%%%%%%%%%%%%%%
% Please include all figure legends in a separate section at the end of 
% the document. If you use \label{} and \ref{} to refer to your figures,
% these can still work even if you comment out the %includegraphics{}
% line. If you refer to figures as "fig. 1" (etc.) manually, the
% legends can also appear simply as paragraphs.
% For submission, please upload the relevant figure files separately to
% Editorial Manager; Editorial Manager should insert them at the end of
% the PDF automatically.
% Figure legends should be concise, though they can be longer than the
% titles of tables.

\section*{Figure legends}

\begin{figure}[h!]
%\includegraphics{horn-of-okapi}
\caption{Figure 
legends should be concise, though they can be longer than the titles of 
tables.}
\label{Fig:OkapiHorn}
\end{figure}

\begin{figure}[h!]
%\includegraphics{elegance}
\caption{In this way, figure legends can be listed at the end of the 
document, with references that work, even if the graphic itself is uploaded
separately.}
\label{Fig:AnotherFigure}
\end{figure}

\subsection*{Online figure legends}

\renewcommand{\thefigure}{A\arabic{figure}}
\setcounter{figure}{0}

\begin{figure}[h!]
%\includegraphics{jumps20m}
\caption{\textit{A}, the quick red fox proceeding to jump 20~m straight 
into the air over not one, but several lazy dogs. \textit{B}, the quick 
red fox landing gracefully despite the skepticism of naysayers.}
\label{Fig:Jumps}
\end{figure}

\begin{figure}[h!]
%\includegraphics{jumps-nr-okapi}
\caption{The quicker the red fox jumps, the likelier it is to land near 
an okapi. For further details, see \citet{LemKapEx07}.}
\label{Fig:JumpsOk}
\end{figure}

\renewcommand{\thefigure}{B\arabic{figure}}
\setcounter{figure}{0}

\end{document}
